% Erstellt von Benjamin Thomitzni
% Version 1.0, 18.06.17
% 
% Für die Erstellung des Dokuments wurde eine Reihe an LaTeX Paketen verwendet. Diese sind unter der LPPL (http://www.latex-project.org/lppl.txt) Lizenz veröffentlich, falls nicht anders genannt.
%
%Das Splittermond Rollenspiel wird vom Uhrwerk-Verlag (http://www.uhrwerk-verlag.de/) vertrieben.
%
%Der Code compiliert mit pdflatex (https://www.ctan.org/pkg/pdftex)

\documentclass[12pt, a4paper, twoside, openany]{book}

\usepackage[utf8x]{inputenc} %https://www.ctan.org/pkg/inputenc
\usepackage{german} %Deutsche Sprache, https://www.ctan.org/pkg/german
\usepackage[]{graphicx} %Zusatzfunktionen für Graphiken, remove draft for release, https://www.ctan.org/pkg/graphicx
\usepackage{blindtext} %für lorem ipsum blindtext, https://www.ctan.org/pkg/blindtext
\usepackage{multicol} %für zweispaltig etc., https://www.ctan.org/pkg/multicol?lang=en
\usepackage[a4paper]{geometry} %Abstände innerhalb der Seiten, https://www.ctan.org/pkg/geometry
\geometry{
        a4paper,
        top=30mm,
        inner=13mm, 
        outer=32mm,
        footskip=10mm
}
\usepackage{float} %Float Optionen für Graphiken, https://www.ctan.org/pkg/float
%Header ------
\usepackage{fancyhdr} %Paket zum Einstellen des Headers, https://www.ctan.org/pkg/fancyhdr
\fancypagestyle{plain}{
        \fancyhead{}
        \fancyfoot{}
        \fancyfoot[RO, LE]{\color{white} \Huge  \textbf{\thepage}} %Fußnote, left even, right odd auf alternierenden Seiten Seitenzahl wechseln
        \fancyfootoffset[RO, LE]{50pt}
        \renewcommand{\headrulewidth}{0pt} % Ausschalten der Headerlinie
        \renewcommand{\footrulewidth}{0pt}
}
\pagestyle{plain}
%----- Farben
\usepackage{anyfontsize} %Größere/Kleinere Zeichen, https://www.ctan.org/pkg/anyfontsize
\usepackage[table]{xcolor} %Farben definierne können, table für rowcolor, https://www.ctan.org/pkg/xcolor
\definecolor{spmblue}{HTML}{25408F}%Überrschriftenfarbe
\definecolor{kastenblau}{HTML}{BADDF3}
\definecolor{kastenumrandung}{HTML}{5E9BD0}
\definecolor{spmback}{HTML}{ECF0F3}%Hintergrundfarbei
\definecolor{spmtabelle}{HTML}{ABE1FA}
\pagecolor{spmback}
%---- Veränderungen an den Überschriften
\usepackage{sectsty} %Kontrollieren von Chapter, Section Positon, https://www.ctan.org/pkg/sectsty

\allsectionsfont{\sffamily \centering} %Alle Überschriften zentrieren
\chapterfont{\sffamily \centering \color{spmblue}}
\subsectionfont{\sffamily \color{spmblue}\centering}
\subsubsectionfont{\sffamily \color{black}}

%---- Geometry paket um die margins einzustellen
\usepackage{authoraftertitle}%title, author und date einstellen und immer abrufen können, https://www.ctan.org/pkg/authoraftertitle, Public Domain Software Lizenz

%--- boxes
\usepackage{tcolorbox} %Farbige Box, https://www.ctan.org/pkg/tcolorbox
\usepackage{tikz} %Tikz, um zu zeichnen, https://www.ctan.org/pkg/pgf?lang=en, GNU Lizenz
\tcbuselibrary{skins}
%---- bessere auzählung
\usepackage{enumitem} %https://www.ctan.org/pkg/enumitem
\setlist{nosep} %oder \setlist{noitemsep}

%---- Metadaten
\title{Titel des Abenteuers}
\author{Autor}
\date{datum}
%----------------------
%Hintergrundeinstellungen
\usepackage{background}%Hintergründe einstellen, https://www.ctan.org/pkg/background
\backgroundsetup{
        scale=1,
        opacity=1,
        angle=0,
        contents={%Um gerade, ungerade zahl zu wechseln
                \ifodd\value{page}
                        \includegraphics[width=\paperwidth]{bilder/seite-rechts.png}
                \else
                        \includegraphics[width=\paperwidth]{bilder/seite-links.png}
                \fi
        }
}
%------------
\usepackage{lmodern}%Schrift, https://www.ctan.org/tex-archive/info/lmodern

%------ Margin für Chapter etc. oben verkleinern
\usepackage{etoolbox} % https://www.ctan.org/pkg/etoolbox
\makeatletter
\patchcmd{\@makechapterhead}{50\p@}{1pt}{}{}
\patchcmd{\@makeschapterhead}{50\p@}{1pt}{}{}
%\makeatother
%\patchcmd{\chapter}{\thispagestyle{plain}}{\thispagestyle{fancy}}{}{}
\makeatother
%----- Defintion der Splittermond Box
\newtcolorbox{spmbox}[2][]{
        boxsep=8mm, 
        width=0.5\textwidth, 
        top=1mm, 
        bottom=1mm, 
        enhanced, 
        frame hidden, 
        interior style image=bilder/Kasten.png
}

%--------------------------------Doppelseite Grapfik Fix------------
\usepackage{adjustbox}
\usepackage{afterpage}
\usepackage{placeins}

% For the \muemoir` class remove the following two packages.
% This class already provide the functionality of both
\usepackage{caption}
\usepackage[strict]{changepage}
%%%

\setcounter{totalnumber}{1}
\setcounter{topnumber}{1}
\setcounter{bottomnumber}{1}
\renewcommand{\topfraction}{.99}
\renewcommand{\bottomfraction}{.99}
\renewcommand{\textfraction}{.01}

\makeatletter
\newcommand*{\twopagepicture}[4]{%
        \checkoddpage
        \ifoddpage
                \expandafter\@firstofone
        \else
                \expandafter\afterpage
        \fi
        {\afterpage{%
                        \if #1t%
                                \if #2p%
                                        \thispagestyle{empty}%
                                        \afterpage{\thispagestyle{empty}}%
                                \fi
                        \fi
                        \begin{figure}[#1]
                                \if #2p%
                                        \if #1t%
                                                \vspace*{-\dimexpr1in+\voffset+\topmargin+\headheight+\headsep\relax}%
                                        \fi
                                \fi
                                \if #1b%
                                        \caption{#4}%
                                \fi
                                \makebox[\textwidth][l]{%
                                        \if #2p\relax
                                                \let\mywidth\paperwidth
                                                \hskip-\dimexpr1in+\hoffset+\evensidemargin\relax
                                        \else
                                                \let\mywidth\linewidth
                                        \fi
                                \adjustbox{trim=0 0 {.5\width} 0,clip}{\includegraphics[width=2\mywidth]{#3}}}%
                                \if #1b\else
                                        \caption{#4}%
                                \fi
                                \if #2p%
                                        \if #1b%
                                                \vspace*{-\dimexpr\paperheight-\textheight-1in-\voffset-\topmargin-\headheight-\headsep\relax}%
                                        \fi
                                \fi
                        \end{figure}%
                        \begin{figure}[#1]
                                \if #2p%
                                        \if #1t%
                                                \vspace*{-\dimexpr1in+\voffset+\topmargin+\headheight+\headsep\relax}%
                                        \fi
                                \fi
                                \makebox[\textwidth][l]{%
                                        \if #2p%
                                                \let\mywidth\paperwidth
                                                \hskip-\dimexpr1in+\hoffset+\oddsidemargin\relax
                                        \else
                                                \let\mywidth\linewidth
                                        \fi
                                \adjustbox{trim={.5\width} 0 0 0,clip}{\includegraphics[width=2\mywidth]{#3}}}%
                                \if #2p%
                                        \if #1b%
                                                \vspace*{-\dimexpr\paperheight-\textheight-1in-\voffset-\topmargin-\headheight-\headsep\relax}%
                                        \fi
                                \fi
                        \end{figure}%
        }}%
}
\makeatother
%--------------------------------------------------------

%------------------------------------------------
\begin{document}
\pagenumbering{gobble}%keine Nummerierung der Seite für die Titelseite und Impressum
\sffamily %Sans Serif Schriftart
\begin{titlepage}
    %Hintergrundbild für das Titelblatt
        \tikz[remember picture,overlay] \node[opacity=1,inner sep=0pt] at (current page.center){\includegraphics[height=\paperheight]{bilder/hintergrund.jpg}};
        %Balken auf der Seite des Titelblattes
        \tikz[remember picture,overlay] \node[opacity=1,inner sep=0pt] at (current page.center){\includegraphics[width=\paperwidth,height=\paperheight]{bilder/titelbild-balken.png}};

        {
                \begin{center}
                        \color{white}
                        \fontsize{50}{60}
                        \selectfont
                        \textbf{\MyTitle}
                \end{center}
        }%Titelseite Block ende
    %Splittermond Fan Logo
        \begin{figure}[b]
                \centering
                \includegraphics[scale=0.8]{bilder/Splittermond-Logo_fan_v2.png}
        \end{figure}
\end{titlepage}
\clearpage
{%Impressum und toc Seite

        {%Titel
                \begin{center}
                        \color{spmblue}
                        \fontsize{40}{50}
                        \selectfont
                        \textbf{\MyTitle}
                \end{center}
        }%Titel
    %Splittermond Fan Logo
        \begin{figure}[t]
                \centering
                \includegraphics[scale=0.8]{bilder/Splittermond-Logo_fan_v2.png}
        \end{figure}

        \vfill
        \section*{Impressum}%* Versionen  da keine Nummern gewollt werden
        \addcontentsline{toc}{chapter}{Impressum}%Um zu Inhaltsverzeichnis hinzuzufügen
        \begin{center}
                Splittermond wird herausgegeben vom \textit{Uhrwerk}-Verlag.\\
                Bei diesem Werk handelt es sich um inoffizielles Fanmaterial.        
        \end{center}
        \subsection*{Autor}
        \begin{center} 
                \MyAuthor
        \end{center}
        \subsection*{Illustrationen}
        \begin{center}
                Max Mustermann        
        \end{center}

        \subsection*{Layout}
        \begin{center}
                Erstellt von Benjamin Thomitzni mit \LaTeX~und compiliert mit \textit{pdflatex}.\\
                Inspririert von der Fanwerk Layoutvorlage von Daniel Bruxmeier für Word. \\         
        \end{center}
        %\subsection*{Lizenz}%Alte Lizenz als noch Graphiken von Daniel Bruxmeier verwendet wurden.
        %\begin{center}
        %        \tiny
        %        Dieses Layout steht unter der \textit{Creative-Commens}-Lizenz. \\
        %        \includegraphics[scale=1]{bilder/Splittermond_Fanpaket/CC-BY.png}\\
        %        Dies umfasst ausdrücklich nicht die eigentlichen Inhalte des Dokuments wie Texte oder zusätzliche Illustrationen. \\
        %        Bei Nutzung diess Layout bitte wenn möglich das endgültige Werk ebenfalls unter einer \textit{Creatives-Commens}-Lizenz stellen. 
        %\end{center}
}
%---------- Inhaltsverzeichnis % entfernen um einzufügen
%        \tableofcontents
%----------
\newpage
\pagenumbering{arabic}%Nummerierung startet hier
\chapter*{\MyTitle}%Chapter*, Section* etc. bedeutet dass sie nicht zur Inhaltsangabe hinzugefügt werden. Deswegen die nächste Zeile
\addcontentsline{toc}{chapter}{\MyTitle}
\begin{multicols}{2}
        \subsubsection*{von \MyAuthor}
        Blindtext blabla
%----------------------Box Code        
        \begin{spmbox}
                {\begin{center}
                                \color{spmblue}
                                \Large
                                \textbf{\MyTitle}
                \end{center}}
                Zusammenfassung des Abenteuers. Text. Blablabla. \\
                \textbf{Region:} Wo auch immer\\
                \textbf{Schauplätze:} In einer Stadt\\
                \textbf{Erfahrung:} Heldengrad X\\
                \textbf{Zeitraum:} 2-999 Tage\\
        \end{spmbox}
%---------------------Kleine Überschriften
        \subsection*{Hintergrund}
        \Blindtext[1][1]
        \subsection*{Handlung}
        \Blindtext[1][1]
        \subsection*{Auswahl der Abenteuerer}
        Irgendwelche Idioten die sterben wollen.             
\end{multicols}
%-------------------- Neuer Abschnitt
\section*{Einstieg: Wer, Wie, Wo, Was, Wie, Hä?}
\addcontentsline{toc}{section}{Einstieg: Wer, Wie, Wo, Was, Wie, Hä?}

\begin{multicols}{2} %Begin zweizweilises Layout
        \Blindtext[1][1]
%--------------------Auflistung        
        \begin{enumerate}[label=\textbullet]
                \item Ein Item der Liste
                \item Ein Zweites Item auf der Liste
        \end{enumerate}

        \subsection*{Nummer Eins}
        \Blindtext[1][1]
        \subsubsection*{Unterüberschrift}
        \Blindtext[1][1]
        {%---- Bild einfügen direkt im Text, da float umgebungen in multicol Probleme machen. 
                \begin{center}
                        \includegraphics[width=0.5\textwidth]{bilder/foo.png}
                \end{center}
        }
\end{multicols}
%--------- Bild einfügen außerhalb des textes
\begin{figure}[H] %H bedeutet "Hier" platzieren
        \centering
        \includegraphics[width=.6\textwidth]{bilder/foo.png}
        \label{fig:piel}
\end{figure}

\begin{multicols}{2}
        \Blindtext[1][1]
%---------------- Gegner Block bis unten
        \vbox{%Gegner Block
                {
                        \vspace{5mm}
                        \Large
                        \textbf{Beispielgegner}
                }\\
                \rule{0.5\textwidth}{0.2em}\\%Querstrich
                \resizebox{0.5\textwidth}{!}{%Automatisches Skalieren der Tabelle
                        \begin{tabular}{c|c|c|c|c|c|c|c}
                                \rowcolor{spmtabelle}
                                \textbf{AUS} & \textbf{BEW} & \textbf{INT} & \textbf{KON} & \textbf{MYS} & \textbf{STÄ} & \textbf{VER} & \textbf{WIL} \\
                                1            & 1            & 1            & 1            & 1            & 1            & 1            & 1            \\
                                \rowcolor{spmtabelle}
                                \textbf{GK}  & \textbf{GSW} & \textbf{LP}  & \textbf{FO}  & \textbf{VTD} & \textbf{SR}  & \textbf{KW}  & \textbf{GW}  \\
                                1            & 1            & 1            & 1            & 1            & 1            & 1            & 1           
                \end{tabular}}\\ 
                \resizebox{0.5\textwidth}{!}{
                        \begin{tabular}{l|c|c|c|c|c}
                                \rowcolor{spmtabelle}
                                \textbf{Waffe} & \textbf{Wertf} & \textbf{Schaden} & \textbf{WGS} & \textbf{INI} & \textbf{Merkmale} \\
                                Körper         &            9   & 1W6              & 6 Ticks      & 7-1W6        & \\ 

                        \end{tabular}
                }%Resizebox
                \vspace{2mm}\\%Abstand
                \textbf{Typus:} Humanoid \\
                \textbf{Monstergrad:} 0/0 \\
                \textbf{Fertigkeiten:} Absolut Nix 1, Nix 2 \\
                \textbf{Meisterschaften:} Bäcker (Handwerk) \\
                \textbf{Merkmale:} Dämmersicht \\
                \textbf{Beute:} Lollipop \\
                \rule{0.5\textwidth}{0.2em}\\
                \vspace{5mm}
        }
%------------------ Gegner Block Ende
\end{multicols}    

%------------ Doppelseite Graphik Dirty Hack
%\newpage
%\begin{figure}
%        \newlengtih{\imagewidth}
%        \settowidth{\imagewidth}{\includegraphics{bilder/hintergrund.jpg}}
%        \newlength{\imageheight}
%        \settoheight{\imageheight}{\includegraphics{bilder/hintergrund.jpg}}
       % \includegraphics[clip, viewport=0px 0px 0.5\imagewidth{} \imageheight{}, width=\paperwidth, height=\paperheight]{bilder/hintergrund.jpg}
%\end{figure}
%        \tikz[remember picture, overlay] \node[opacity=1,inner sep=0pt] at (current page.center){\includegraphics[clip, viewport=0px 0px 0.5\imagewidth{} \imageheight{}, width=\paperwidth, height=\paperheight]{bilder/hintergrund.jpg}};
%\newpage 
%        \tikz[remember picture, overlay] \node[opacity=1,inner sep=0pt] at (current page.center){\includegraphics[clip, viewport=0.5\imagewidth{} 0px \imagewidth{} \imageheight{}, width=\paperwidth, height=\paperheight]{bilder/hintergrund.jpg}};
%\newpage

%---------------- Hier weiter mit normalen seiten

%Doppelseite Graphik von stackoverflow
\twopagepicture{t}{p}{bilder/hintergrund.jpg}{Test}
\end{document}

